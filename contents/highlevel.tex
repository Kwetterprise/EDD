\chapter{High Level Design}
\cref{fig:highlevel} contains a diagram of Kwetterprise's architecture.

\begin{tefigure}[H][centering]
    \teincludegraphics[width=.7\textwidth]{fig:highlevel}{High level design.}{figures/highlevel}
\end{tefigure}

This architecture consists of four main parts:
\begin{teitemize}
    \item \textbf{Microservices}\\
    The business logic of Kwetterprise is handled by multiple microservices. Microservices are chosen to seperate responsibilies and functionality.
    \item \textbf{API gateway}\\
    The API gateway is used to dyamically route requests from the front-end (or other API users) to the correct micro-service.
    \item \textbf{Event bus}\\
    Communication between microservices is event-driven. A so called ``event bus'' is responsible for accepting and delivering events.
\end{teitemize}

\section{Service design}
Services are designed with the ``command–query responsibility segregation'' principle (CQRS) in mind. \cref{fig:highlevel-service} shows this archtecture.

\begin{tefigure}[h][centering]
    \teincludegraphics[width=.6\textwidth]{fig:highlevel-service}{High level design of a service.}{figures/service}
\end{tefigure}

The service consists of two parts. The \textit{Command} part handles commands which are invoked by an actor such as tweeting or following a user. These commands result in events being published to the event bus.

The \textit{Query} part of the service is responsible for presenting this data to other components. It processes the events created by the \textit{Command} component and keeps a database with the current state. This database functions as a ``cache'' for query requests like retrieving a user's tweets or followers.

\FloatBarrier%
The reason this cache exists is because otherwise a ``retrieve all followers from this user'' request must fetch all events relating to that user (following, unfollowing, account deletion, etc) and process them to determine the final result. This would put high load on the service the moment a query is executed instead of balancing the load whenever updates are pushed.

Thus, the SPoT (single-point of truth) is the events in the events bus. Note that the ``current state cache'' database can always be recreated by processing all historic events.